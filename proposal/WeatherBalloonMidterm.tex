
%% bare_conf_compsoc.tex
%% V1.4b
%% 2015/08/26
%% by Michael Shell
%% See:
%% http://www.michaelshell.org/
%% for current contact information.
%%
%% This is a skeleton file demonstrating the use of IEEEtran.cls
%% (requires IEEEtran.cls version 1.8b or later) with an IEEE Computer
%% Society conference paper.
%%
%% Support sites:
%% http://www.michaelshell.org/tex/ieeetran/
%% http://www.ctan.org/pkg/ieeetran
%% and
%% http://www.ieee.org/

%%*************************************************************************
%% Legal Notice:
%% This code is offered as-is without any warranty either expressed or
%% implied; without even the implied warranty of MERCHANTABILITY or
%% FITNESS FOR A PARTICULAR PURPOSE! 
%% User assumes all risk.
%% In no event shall the IEEE or any contributor to this code be liable for
%% any damages or losses, including, but not limited to, incidental,
%% consequential, or any other damages, resulting from the use or misuse
%% of any information contained here.
%%
%% All comments are the opinions of their respective authors and are not
%% necessarily endorsed by the IEEE.
%%
%% This work is distributed under the LaTeX Project Public License (LPPL)
%% ( http://www.latex-project.org/ ) version 1.3, and may be freely used,
%% distributed and modified. A copy of the LPPL, version 1.3, is included
%% in the base LaTeX documentation of all distributions of LaTeX released
%% 2003/12/01 or later.
%% Retain all contribution notices and credits.
%% ** Modified files should be clearly indicated as such, including  **
%% ** renaming them and changing author support contact information. **
%%*************************************************************************


% *** Authors should verify (and, if needed, correct) their LaTeX system  ***
% *** with the testflow diagnostic prior to trusting their LaTeX platform ***
% *** with production work. The IEEE's font choices and paper sizes can   ***
% *** trigger bugs that do not appear when using other class files.       ***                          ***
% The testflow support page is at:
% http://www.michaelshell.org/tex/testflow/



\documentclass[conference,compsoc]{IEEEtran}
% Some/most Computer Society conferences require the compsoc mode option,
% but others may want the standard conference format.
%
% If IEEEtran.cls has not been installed into the LaTeX system files,
% manually specify the path to it like:
% \documentclass[conference,compsoc]{../sty/IEEEtran}





% Some very useful LaTeX packages include:
% (uncomment the ones you want to load)


% *** MISC UTILITY PACKAGES ***
%
%\usepackage{ifpdf}
% Heiko Oberdiek's ifpdf.sty is very useful if you need conditional
% compilation based on whether the output is pdf or dvi.
% usage:
% \ifpdf
%   % pdf code
% \else
%   % dvi code
% \fi
% The latest version of ifpdf.sty can be obtained from:
% http://www.ctan.org/pkg/ifpdf
% Also, note that IEEEtran.cls V1.7 and later provides a builtin
% \ifCLASSINFOpdf conditional that works the same way.
% When switching from latex to pdflatex and vice-versa, the compiler may
% have to be run twice to clear warning/error messages.






% *** CITATION PACKAGES ***
%
\ifCLASSOPTIONcompsoc
  % IEEE Computer Society needs nocompress option
  % requires cite.sty v4.0 or later (November 2003)
  \usepackage[nocompress]{cite}
\else
  % normal IEEE
  \usepackage{cite}
\fi
% cite.sty was written by Donald Arseneau
% V1.6 and later of IEEEtran pre-defines the format of the cite.sty package
% \cite{} output to follow that of the IEEE. Loading the cite package will
% result in citation numbers being automatically sorted and properly
% "compressed/ranged". e.g., [1], [9], [2], [7], [5], [6] without using
% cite.sty will become [1], [2], [5]--[7], [9] using cite.sty. cite.sty's
% \cite will automatically add leading space, if needed. Use cite.sty's
% noadjust option (cite.sty V3.8 and later) if you want to turn this off
% such as if a citation ever needs to be enclosed in parenthesis.
% cite.sty is already installed on most LaTeX systems. Be sure and use
% version 5.0 (2009-03-20) and later if using hyperref.sty.
% The latest version can be obtained at:
% http://www.ctan.org/pkg/cite
% The documentation is contained in the cite.sty file itself.
%
% Note that some packages require special options to format as the Computer
% Society requires. In particular, Computer Society  papers do not use
% compressed citation ranges as is done in typical IEEE papers
% (e.g., [1]-[4]). Instead, they list every citation separately in order
% (e.g., [1], [2], [3], [4]). To get the latter we need to load the cite
% package with the nocompress option which is supported by cite.sty v4.0
% and later.





% *** GRAPHICS RELATED PACKAGES ***
%
\ifCLASSINFOpdf
   \usepackage[pdftex]{graphicx}
  % declare the path(s) where your graphic files are
   %\graphicspath{{../pdf/}{../jpeg/}}
  % and their extensions so you won't have to specify these with
  % every instance of \includegraphics
   \DeclareGraphicsExtensions{.pdf,.jpeg,.png}
\else
  % or other class option (dvipsone, dvipdf, if not using dvips). graphicx
  % will default to the driver specified in the system graphics.cfg if no
  % driver is specified.
  % \usepackage[dvips]{graphicx}
  % declare the path(s) where your graphic files are
  % \graphicspath{{../eps/}}
  % and their extensions so you won't have to specify these with
  % every instance of \includegraphics
  % \DeclareGraphicsExtensions{.eps}
\fi
% graphicx was written by David Carlisle and Sebastian Rahtz. It is
% required if you want graphics, photos, etc. graphicx.sty is already
% installed on most LaTeX systems. The latest version and documentation
% can be obtained at: 
% http://www.ctan.org/pkg/graphicx
% Another good source of documentation is "Using Imported Graphics in
% LaTeX2e" by Keith Reckdahl which can be found at:
% http://www.ctan.org/pkg/epslatex
%
% latex, and pdflatex in dvi mode, support graphics in encapsulated
% postscript (.eps) format. pdflatex in pdf mode supports graphics
% in .pdf, .jpeg, .png and .mps (metapost) formats. Users should ensure
% that all non-photo figures use a vector format (.eps, .pdf, .mps) and
% not a bitmapped formats (.jpeg, .png). The IEEE frowns on bitmapped formats
% which can result in "jaggedy"/blurry rendering of lines and letters as
% well as large increases in file sizes.
%
% You can find documentation about the pdfTeX application at:
% http://www.tug.org/applications/pdftex





% *** MATH PACKAGES ***
%
%\usepackage{amsmath}
% A popular package from the American Mathematical Society that provides
% many useful and powerful commands for dealing with mathematics.
%
% Note that the amsmath package sets \interdisplaylinepenalty to 10000
% thus preventing page breaks from occurring within multiline equations. Use:
%\interdisplaylinepenalty=2500
% after loading amsmath to restore such page breaks as IEEEtran.cls normally
% does. amsmath.sty is already installed on most LaTeX systems. The latest
% version and documentation can be obtained at:
% http://www.ctan.org/pkg/amsmath





% *** SPECIALIZED LIST PACKAGES ***
%
%\usepackage{algorithmic}
% algorithmic.sty was written by Peter Williams and Rogerio Brito.
% This package provides an algorithmic environment fo describing algorithms.
% You can use the algorithmic environment in-text or within a figure
% environment to provide for a floating algorithm. Do NOT use the algorithm
% floating environment provided by algorithm.sty (by the same authors) or
% algorithm2e.sty (by Christophe Fiorio) as the IEEE does not use dedicated
% algorithm float types and packages that provide these will not provide
% correct IEEE style captions. The latest version and documentation of
% algorithmic.sty can be obtained at:
% http://www.ctan.org/pkg/algorithms
% Also of interest may be the (relatively newer and more customizable)
% algorithmicx.sty package by Szasz Janos:
% http://www.ctan.org/pkg/algorithmicx




% *** ALIGNMENT PACKAGES ***
%
%\usepackage{array}
% Frank Mittelbach's and David Carlisle's array.sty patches and improves
% the standard LaTeX2e array and tabular environments to provide better
% appearance and additional user controls. As the default LaTeX2e table
% generation code is lacking to the point of almost being broken with
% respect to the quality of the end results, all users are strongly
% advised to use an enhanced (at the very least that provided by array.sty)
% set of table tools. array.sty is already installed on most systems. The
% latest version and documentation can be obtained at:
% http://www.ctan.org/pkg/array


% IEEEtran contains the IEEEeqnarray family of commands that can be used to
% generate multiline equations as well as matrices, tables, etc., of high
% quality.




% *** SUBFIGURE PACKAGES ***
%\ifCLASSOPTIONcompsoc
%  \usepackage[caption=false,font=footnotesize,labelfont=sf,textfont=sf]{subfig}
%\else
%  \usepackage[caption=false,font=footnotesize]{subfig}
%\fi
% subfig.sty, written by Steven Douglas Cochran, is the modern replacement
% for subfigure.sty, the latter of which is no longer maintained and is
% incompatible with some LaTeX packages including fixltx2e. However,
% subfig.sty requires and automatically loads Axel Sommerfeldt's caption.sty
% which will override IEEEtran.cls' handling of captions and this will result
% in non-IEEE style figure/table captions. To prevent this problem, be sure
% and invoke subfig.sty's "caption=false" package option (available since
% subfig.sty version 1.3, 2005/06/28) as this is will preserve IEEEtran.cls
% handling of captions.
% Note that the Computer Society format requires a sans serif font rather
% than the serif font used in traditional IEEE formatting and thus the need
% to invoke different subfig.sty package options depending on whether
% compsoc mode has been enabled.
%
% The latest version and documentation of subfig.sty can be obtained at:
% http://www.ctan.org/pkg/subfig




% *** FLOAT PACKAGES ***
%
%\usepackage{fixltx2e}
% fixltx2e, the successor to the earlier fix2col.sty, was written by
% Frank Mittelbach and David Carlisle. This package corrects a few problems
% in the LaTeX2e kernel, the most notable of which is that in current
% LaTeX2e releases, the ordering of single and double column floats is not
% guaranteed to be preserved. Thus, an unpatched LaTeX2e can allow a
% single column figure to be placed prior to an earlier double column
% figure.
% Be aware that LaTeX2e kernels dated 2015 and later have fixltx2e.sty's
% corrections already built into the system in which case a warning will
% be issued if an attempt is made to load fixltx2e.sty as it is no longer
% needed.
% The latest version and documentation can be found at:
% http://www.ctan.org/pkg/fixltx2e


%\usepackage{stfloats}
% stfloats.sty was written by Sigitas Tolusis. This package gives LaTeX2e
% the ability to do double column floats at the bottom of the page as well
% as the top. (e.g., "\begin{figure*}[!b]" is not normally possible in
% LaTeX2e). It also provides a command:
%\fnbelowfloat
% to enable the placement of footnotes below bottom floats (the standard
% LaTeX2e kernel puts them above bottom floats). This is an invasive package
% which rewrites many portions of the LaTeX2e float routines. It may not work
% with other packages that modify the LaTeX2e float routines. The latest
% version and documentation can be obtained at:
% http://www.ctan.org/pkg/stfloats
% Do not use the stfloats baselinefloat ability as the IEEE does not allow
% \baselineskip to stretch. Authors submitting work to the IEEE should note
% that the IEEE rarely uses double column equations and that authors should try
% to avoid such use. Do not be tempted to use the cuted.sty or midfloat.sty
% packages (also by Sigitas Tolusis) as the IEEE does not format its papers in
% such ways.
% Do not attempt to use stfloats with fixltx2e as they are incompatible.
% Instead, use Morten Hogholm'a dblfloatfix which combines the features
% of both fixltx2e and stfloats:
%
% \usepackage{dblfloatfix}
% The latest version can be found at:
% http://www.ctan.org/pkg/dblfloatfix




% *** PDF, URL AND HYPERLINK PACKAGES ***
%
%\usepackage{url}
% url.sty was written by Donald Arseneau. It provides better support for
% handling and breaking URLs. url.sty is already installed on most LaTeX
% systems. The latest version and documentation can be obtained at:
% http://www.ctan.org/pkg/url
% Basically, \url{my_url_here}.




% *** Do not adjust lengths that control margins, column widths, etc. ***
% *** Do not use packages that alter fonts (such as pslatex).         ***
% There should be no need to do such things with IEEEtran.cls V1.6 and later.
% (Unless specifically asked to do so by the journal or conference you plan
% to submit to, of course. )


% correct bad hyphenation here
\hyphenation{op-tical net-works semi-conduc-tor}


\begin{document}
%
% paper title
% Titles are generally capitalized except for words such as a, an, and, as,
% at, but, by, for, in, nor, of, on, or, the, to and up, which are usually
% not capitalized unless they are the first or last word of the title.
% Linebreaks \\ can be used within to get better formatting as desired.
% Do not put math or special symbols in the title.
\title{Weather Balloon Project Midterm Report}


% author names and affiliations
% use a multiple column layout for up to three different
% affiliations
\author{\IEEEauthorblockN{Clayton Davis}
\IEEEauthorblockA{Department of Electrical Engineering \\and Computer Science\\
The University of Tennessee, Knoxville\\
Knoxville, TN 37996\\
Email: cdavi151@vols.utk.edu}
\and
\IEEEauthorblockN{Sunay Bhat}
\IEEEauthorblockA{Department of Electrical Engineering \\and Computer Science\\
The University of Tennessee, Knoxville\\
Knoxville, TN 37996\\
Email: sbhat3@vols.utk.edu}
\and
\IEEEauthorblockN{Thomas Turner}
\IEEEauthorblockA{Department of Electrical Engineering \\and Computer Science\\
The University of Tennessee, Knoxville\\
Knoxville, TN 37996\\
Email: tturne28@vols.utk.edu}}

% conference papers do not typically use \thanks and this command
% is locked out in conference mode. If really needed, such as for
% the acknowledgment of grants, issue a \IEEEoverridecommandlockouts
% after \documentclass

% for over three affiliations, or if they all won't fit within the width
% of the page (and note that there is less available width in this regard for
% compsoc conferences compared to traditional conferences), use this
% alternative format:
% 
%\author{\IEEEauthorblockN{Michael Shell\IEEEauthorrefmark{1},
%Homer Simpson\IEEEauthorrefmark{2},
%James Kirk\IEEEauthorrefmark{3}, 
%Montgomery Scott\IEEEauthorrefmark{3} and
%Eldon Tyrell\IEEEauthorrefmark{4}}
%\IEEEauthorblockA{\IEEEauthorrefmark{1}School of Electrical and Computer Engineering\\
%Georgia Institute of Technology,
%Atlanta, Georgia 30332--0250\\ Email: see http://www.michaelshell.org/contact.html}
%\IEEEauthorblockA{\IEEEauthorrefmark{2}Twentieth Century Fox, Springfield, USA\\
%Email: homer@thesimpsons.com}
%\IEEEauthorblockA{\IEEEauthorrefmark{3}Starfleet Academy, San Francisco, California 96678-2391\\
%Telephone: (800) 555--1212, Fax: (888) 555--1212}
%\IEEEauthorblockA{\IEEEauthorrefmark{4}Tyrell Inc., 123 Replicant Street, Los Angeles, California 90210--4321}}




% use for special paper notices
%\IEEEspecialpapernotice{(Invited Paper)}




% make the title area
\maketitle

% As a general rule, do not put math, special symbols or citations
% in the abstract
\begin{abstract}
Develop a low-cost airborne sensor platform, or radiosonde, with GPS and telemetry reporting 
that focuses on specific environmental health standards such as air quality and pollutant levels.
\end{abstract}

% no keywords




% For peer review papers, you can put extra information on the cover
% page as needed:
% \ifCLASSOPTIONpeerreview
% \begin{center} \bfseries EDICS Category: 3-BBND \end{center}
% \fi
%
% For peerreview papers, this IEEEtran command inserts a page break and
% creates the second title. It will be ignored for other modes.
\IEEEpeerreviewmaketitle



\section{Introduction}
% no \IEEEPARstart

\subsection{Aim}
In this document, we articulate a design to create a low-cost weather balloon with sensor data collection
 and telemetry/telecommand via radio. Its primary purpose as a sensor system is to take frequent data 
 measurements of specific environmental/health pollutants such that a general understanding of 
 environmental conditions and safety can be determined by the user. Common air pollutants, such as 
 carbon dioxide, ozone, nitrogen oxides, and lead will be our primary focus, although some other particulate 
 matter might be considered as the design evolves. Ultimately, users will be able to process data based on 
 relayed GPS coordinates, altitude, and time giving an insightful picture into the local environment’s safety 
 and health. Frequent and repetitive use of such a low-cost device could also be used for long term 
 research by academic or health group interests. 


\subsection{Purpose}
The primary goal of our proposed design is to build a weather balloon device that could contribute value to groups and users concerned with environmental health standards. As the scope and understanding of environmental challenges facing society has increased, the day to day health hazards and risks people face will continue to increase and be a cause for greater concern as well. Our intention is to contribute to low-cost sensor detection of such hazards by taking advantage of increasingly small and efficient embedded system components. Such a device could then be customized for very specific uses by simply interchanging the specific sensor/pollutant detection devices to tailor to a variety of needs.  Envisioning an entire platform for personal environmental health detection devices, our proposed design is ideally a gateway to serve many potential needs.
  
\subsection{Goals}
  Upon completion, we expect that this project will provide us with significant interdisciplinary engineering experience.  The constraints on the system such as weight, power, and computational expense will broaden our expertise beyond solving ordinary hardware and software issues.  Through repeated flights, we expect to model day to day changes in the local atmosphere, and draw conclusions about the relationship of weather, pollution and the environment.

\section{Requirements}
In addition to the primary purpose of sensing, collecting, and relaying atmospheric data, the system must meet significant physical constraints in order to be successfully launched and retrieved from its operational environment.  
\subsection{Sensing}
The system should be equipped with sensors that produce meaningful and actionable atmospheric data.  Typical radiosondes are equipped with several types of gas, particulate, and meteorological sensors.  We aim to replicate this suite as closely as possible within a limited financial, weight, and power budget.  Sensor and location data need not be produced more often than once per minute, which is believed to be sufficient for atmospheric sounding. A secondary consideration is to capture images of the environment to correlate anomalous readings with potential offenders. 

\subsection{Mission Profile}
A typical radiosonde mission lasts for several hours at high altitude, and the system may drift tens to hundreds of miles in that time.  Accurate location reporting is necessary to correlate the atmospheric data with a 3D physical location, as well as retrieve the system at the end of the mission.  Thermal and environmental protection is necessary to ensure proper operation of electronic components and sensors.  

\begin{figure}[!t]
\centering
\includegraphics[width=3in]{balloon_keys.jpg}
% where an .eps filename suffix will be assumed under latex, 
% and a .pdf suffix will be assumed for pdflatex; or what has been declared
% via \DeclareGraphicsExtensions.
\caption{Testing balloon lift capacity.}
\label{fig_sim}
\end{figure}

\subsection{Lift}
The lifting mechanism must deliver the radiosonde to the target altitude for the duration of the mission, and ideally will allow the device to return to ground level with as little damage as possible.  Helium balloons are the obvious choice, although sounding is also conducted using air-dropped and rocket-propelled payloads.  Using a passive lift mechanism places severe constraints on the mass of the payload; analysis is required to determine the feasible upper bound on mass.

\subsection{Telemetry}
The very real risk of payload loss requires that telemetry data be sent to the operator at regular intervals. The outputs of every sensor as well as location, altitude, and system condition should be transmitted.

\subsection{Data Collection}
While the intention is to provide real-time telemetry to the operator, an onboard storage system should be used to cache data in the event of communication failure.  

\subsection{Telecommand}
While the radiosonde is designed to be autonomous in flight, it should be possible for the operator to send commands to the radiosonde in order to control various subsystems or adjust the sensing platform.

\subsection{Power}
Sufficient power should be available to meet the mission objectives and mass constraints.

\subsection{Cost}
Ideally, the radiosonde will use off-the-shelf components at a reasonable price.  However, commercial radiosondes benefit from economies of scale in manufacturing, and the total cost for this system may not be less than a typical unit.

\section{Specification}
Upon further investigation and testing, our team developed a more refined specification for the weather balloon system.

\subsection{Sensing}
The sensors that will be used consist of those that measure environmental conditions (pressure, temperature, etc) and those that measure hazardous gases (carbon dioxide, ozone, nitrogen oxides, etc). The MQ series sensors use a 2-5 V source that is either fixed or alternating between this range to power an internal heater that detects the gases. The sensitivity of the sensor relies on a load-resistor that must be in the range 2 kΩ to 47 kΩ where the lower the resistor value the lower the sensitivity. Some of the sensors will also require additional coding or the use of a transistor to calibrate correctly. The sensors output an analog signal. With a large variety of available sensors, our focus will be to choose those that will be the most efficient, affordable, and capable of handling conditions faced at higher altitudes. 

\subsection{System Control}
A versatile, lightweight, low-power microcontroller platform is central to the functional requirements of the system.  As a sensor platform, the controller must be capable of interfacing with a variety of sensors, both analog and digital, and forwarding the data to the storage and telemetry subsystems. The system controller must be capable of scheduling multiple tasks including data collection and persistence, radio operation, and power management.  Hard real-time performance is not required, but the sensor data should be able to be correlated with a particular time and location within a few seconds.  Sufficient RAM and fixed storage should be available for data and images.

Since the development cycle is short, a familiar, well-supported development environment is preferred.  The efficiency of the program will have to be prioritized within the constraints of limited program space and limited communication bandwidth.

\subsection{Position}
Typical GPS receivers are capable of outputting position and altitude information at rates of 1 to 10 Hz.  Since this exceeds the needs of the system, the GPS receiver can be readily optimized for mass.  The output is typically formatted into NMEA standard sentences which must be decoded by the system controller.

\subsection{Telemetry/Telecommand}
Since the sensor and location data are primarily text-based formats, a radio subsystem capable of direct digital text messaging is preferable.  If an imaging sensor is implemented, a corresponding data format for images must also be supported. Depending on the characteristics of the radio, it may be necessary to add management circuitry to control the radio’s power consumption.  The antennas for both position reception and data transmission should be lightweight and compact.

\subsection{Lift}
The US Naval Academy has launched several experimental payloads with common Mylar party balloons providing lift up to 30k ft and durations of thousands of miles.  While our payload mass will likely exceed theirs, we desire a shorter mission duration and altitude as well as predictable recovery.  

Experimentation with 36-inch diameter balloons yielded an average lifting capacity of 40-45 g at a cost of \$9.95 per balloon.  To manage costs, the mass target of the radiosonde is decided as 160 g or less.

We expect to jettison a portion of the balloon assembly using at the desired conclusion of each flight.  This may be initiated automatically, or via remote command if implemented.  A reduced number of balloons will remain attached to the payload to provide a predictable descent.

\subsection{Payload Chassis}
Ascent to 30,000 feet altitude will present severe cooling stress to the system.  The Naval Academy payloads have used solar thermal regulation (via plastic drink bottles) to avoid the weight of styrofoam and chemical heat packs.  We expect to investigate both solar and chemical thermal regulation, and to choose the option that will best protect the electrical components on landing.

\subsection{Power Supply}
Power to the payload systems is expected to be provided by a lithium-ion (Li-Ion) or lithium-polymer (LiPo) battery pack providing at least 2100 mAh capacity.  The sensors, radio and system controller may require different voltages and multiple sources or voltage conversion may be required.

\section{System Architecture}
After compiling the complete set of requirements for the system, our team investigated various components to meet the requirements.

\begin{figure}[!t]
\centering
\includegraphics[width=3.5in]{system_diagram.png}
% where an .eps filename suffix will be assumed under latex, 
% and a .pdf suffix will be assumed for pdflatex; or what has been declared
% via \DeclareGraphicsExtensions.
\caption{System block diagram.}
\label{fig_sim}
\end{figure}

\begin{figure}[!t]
\centering
\includegraphics[width=3.5in]{hardware_diagram.png}
% where an .eps filename suffix will be assumed under latex, 
% and a .pdf suffix will be assumed for pdflatex; or what has been declared
% via \DeclareGraphicsExtensions.
\caption{Finalized hardware diagram.}
\label{fig_sim}
\end{figure}

\subsection{System Controller}
The Arduino platform and hardware support is well established and supported for prototyping and embedded system design. This makes it an ideal candidate for the system controller. Similarly, there are many sensor/radio “shields” (hardware extensions) that can be added to complete the system design. Commercially-available Arduino boards support multiple I/O interfaces for digital and analog input, and SPI, TTL serial, and I2C buses which should adequately support our hazard and supplementary sensors. The Arduino IDE is built around C/C++ programming languages, with extensive library support for communication protocols and data logging. Most hardware manufacturers provide library code as well, freeing our software design to focus on the control aspects of managing multiple sensors and I/O devices. 

Initially, the Arduino UNO R3, which uses an Atmel ATmega328 microcontroller, was chosen as a suitable platform for development.  The UNO is the most popular of the Arduino boards and nearly every shield and peripheral is compatible with it.  However, other Arduino controllers have less mass and can be operated with 3.3 V logic instead of the 5V required by the UNO. Newer and more capable microcontrollers are also available which retain support for most Arduino libraries.  

For these reasons, the Adafruit Feather M0 “Adalogger” board was chosen for the system controller.  This series contains a 48 MHz Atmel ARM Cortex M0 core, 32K RAM, and 256K flash, as well as an on-board Secure Digital mass storage interface.  The board is designed to be powered by a 3.7V LiPo battery and contains a 100 mA on-board USB-powered charger.  Perhaps most importantly, the board has a desirably small mass of 5.3 g, versus the UNO’s 26 g.  The I/O complement of the Feather M0 is similar to the UNO; 10 analog and 8 digital pins are available in addition to I2C, SPI, and a TTL serial UART.  

Since the Feather M0 uses 3.3 V logic, sensors must either operate at this voltage or use level conversion to be interfaced successfully.  A unity gain amplifier combined with a resistor divider in a 1.7:3.3 ratio will accomplish this if necessary.

\begin{figure}[!t]
\centering
\includegraphics[width=3in]{hamshield.jpg}
% where an .eps filename suffix will be assumed under latex, 
% and a .pdf suffix will be assumed for pdflatex; or what has been declared
% via \DeclareGraphicsExtensions.
\caption{Evaluating HamShield and barometric pressure sensor.}
\label{fig_sim}
\end{figure}

\subsection{Data Transceiver}
Our initial design for communication with the airborne payload was to occur via an onboard VHF/UHF amateur radio transceiver. An Arduino-compatible transceiver, the “HamShield”, was chosen to be evaluated due to its extensive software support and feature set.

The Automatic Packet Reporting System (APRS) is “a packet communications protocol for disseminating live data to everyone on a network in real time”, and would have facilitated the telemetry/telecommand functions.  APRS messages are formatted with the AX.25 protocol, modulated by AFSK, and transmitted on 144.39 MHz. Several formats are available; the telemetry and text message formats are applicable to the radiosonde application. Digital terrestrial repeater stations receive each message and retransmit it at higher power for ease of reception by monitoring stations.  The APRS data is also collected via the Internet by several organizations for later retrieval.  For example, the aprs.fi Web site plots APRS position data using Google Maps (which will be used to track the payload), and aggegates the sensor telemetry for ease of analysis.

The radio would have also accommodated image transmission via slow-scan television (SSTV) protocols.  Amateur radio SSTV is capable of transmitting a 256-line RGB image in approximately 60 seconds over UHF modes.  Downlink reception can be carried out at the operator station using a typical 70 cm capable handheld transceiver and a PC with appropriate decoding software. 

The HamShield is capable of 500 mW ERP, which implies it cannot be powered solely by the Arduino controller board.  A 9V power source was therefore required.  Testing with the HamShield and 9V lithium batteries demonstrated that the signal could not be transmitted at sufficient power to reach the tracking station without causing the controller to reset.  Other undesirable characteristics of the HamShield are an inability to switch off the receiver independently and its use of 5 analog pins on the Arduino.  The support libraries also occupied 36% of program memory and 65% of data RAM, since the AFSK modulation was done in software.  These issues necessitated the rejection of the HamShield.

Since any VHF/UHF transceiver was likely to express similar requirements, an alternative solution was needed. GSM cellular transceivers presented a low-cost, widely tested, and reliable means of meeting the telemetry requirement. The Adafruit FONA 808 is a combined GSM and GPS transceiver module with software support for Arduino control.  The GSM component is a 2G GSM/GPRS transceiver with support for voice, SMS, MMS, and data modes.  All of these modes are accessed through a TTL serial interface, without requiring the Arduino to manage the radio.  This greatly simplifies software interfacing to the module as well as memory usage.  Telemetry will be transmitted via SMS, and images via MMS.  SMS reception will be used to send commands to the system.  The format of messages will be retained from APRS since they provide a simple, unified format for telemetry and commands.

The FONA also requires a 3.7V LiPo battery of at least 500 mAh capacity to be connected at all times while in use.  This is required to accommodate higher current needs during transmission, but also constrains the type of power source.  An external GSM antenna is required, but has a negligible mass penalty of 0.5 g compared to the 12 g VHF antenna used with the HamShield.

\subsection{GPS Receiver}
Prior to discarding the HamShield, a UBlox GP-735 GPS receiver with integrated antenna was chosen for its small dimensions and mass.  However, the FONA 808 contains a GPS receiver as well, making the UBlox redundant.  However, the FONA requires an external passive GPS antenna.  A 15 mm antenna will be used, which is larger than the UBlox integrated antenna and has a 5.5 g mass.  By combining the GSM and GPS transceivers and deleting the HamShield and UBlox, a mass savings of close to 6 g was realized. 

\subsection{Sensors}
The sensors chosen thus far for our design include the MQ-4 (Methane) sensor, MQ-7 (Carbon Monoxide) sensor, and the BMP180 Barometric Pressure Sensor. All three are low mass relative to our design and also low cost making them more ideal. Each have an easily available code library making the primary focus hardware configuration. The BMP180 was chosen because of its low power. It’s also easy to integrate into the system using the I2C interface where we can get accurate measurements of the altitude (which we must give an initial value), temperature, and pressure. The MQ-4 and MQ-7 sensors were chosen because of their low-cost and low-weight qualities along with their compatibility with the arduino and available code library. The sensors are preheated with a voltage of 5 V for a set time then around 1.4 V for another short time (60 seconds). This seems to be a current issue with the design.

\subsection{Camera}
The Adafruit TTL serial camera is capable of capturing 640x480 JPEG images.  These images will be saved to the SD storage and transmitted as MMS images.  The FONA 808 module accepts JPEG images for MMS without conversion to another format, saving time and memory in the system controller.  The camera will be interfaced with the controller using the NewSoftSerial library and digital pins, since the hardware UART will be occupied by the FONA.

\subsection{Power}
The Feather and FONA were both designed to work with 3.7V LiPo batteries, but the sensors and camera both require a 5V rail.  This will be provided by an Adafruit PowerBoost 1000C boost converter, capable of delivering 1000 mA at 5.2 VDC from a single 2500 mAh LiPo battery.  The battery will be directly wired to the Feather and FONA as these modules have their own converters.

If the total current requirement of the complete system is found to exceed the capacity of the battery or boost converter, the sensors and/or FONA will need to be powered off.  The FONA has both an enable line and a command to enter a quiescent state, but the sensors are wired directly to 5V.  By placing a transistor on the ground leg and wiring the base to a digital pin, the current to the sensor can be cut off.

\subsection{Lift Jettison System}
There may be an instance where the radiosonde is malfunctioning or off course, and immediate retrieval is desired.  By jettisoning a portion of the balloons manually, the descent can be partially controlled.  By attaching balloons with fishing line and looping the line around a 10 ohm resistor, the balloons can be separated by passing 5V through the resistor for a few seconds, burning through the line.  A remote SMS command will be used to trigger the jettison, but a hard timeout will also be programmed as a failsafe.

\subsection{Lift System / Chassis}
As noted during the specification phase, each 36-inch balloon will be rated to lift 40 g. The current total mass of the system excluding chassis, resistors, and wire is about 100 g.  With four balloons attached, the radiosonde is expected to rise quickly and easily.  A 0.5 liter soda bottle will serve as the system chassis.  The bottle can be easily reduced in mass, while the top provides a suitable surface for the jettison resistor and GPS antenna.

\section{Remaining Tasks}
With the hardware architecture complete, our team will be turning to the software design.  The library required to interface with the pressure sensor is integrated into our code.  Libraries for the FONA and camera will need to be integrated when the parts arrive.  A schedule of events will need to be created with an estimated energy consumption per cycle, so that the frequency of data capture and telemetry can be verified against available power before the first flight.

We expect to complete the software within the next three weeks, so that the first flight takes place no later than November 5.  If major changes to the architecture must be made after that point, there is still time to receive and integrate components before the due date.  We are confident, however, that no substantial changes will be necessary and that we will be able to conduct multiple successful flights.


% An example of a floating figure using the graphicx package.
% Note that \label must occur AFTER (or within) \caption.
% For figures, \caption should occur after the \includegraphics.
% Note that IEEEtran v1.7 and later has special internal code that
% is designed to preserve the operation of \label within \caption
% even when the captionsoff option is in effect. However, because
% of issues like this, it may be the safest practice to put all your
% \label just after \caption rather than within \caption{}.
%
% Reminder: the "draftcls" or "draftclsnofoot", not "draft", class
% option should be used if it is desired that the figures are to be
% displayed while in draft mode.
%
%\begin{figure}[!t]
%\centering
%\includegraphics[width=2.5in]{myfigure}
% where an .eps filename suffix will be assumed under latex, 
% and a .pdf suffix will be assumed for pdflatex; or what has been declared
% via \DeclareGraphicsExtensions.
%\caption{Simulation results for the network.}
%\label{fig_sim}
%\end{figure}

% Note that the IEEE typically puts floats only at the top, even when this
% results in a large percentage of a column being occupied by floats.


% An example of a double column floating figure using two subfigures.
% (The subfig.sty package must be loaded for this to work.)
% The subfigure \label commands are set within each subfloat command,
% and the \label for the overall figure must come after \caption.
% \hfil is used as a separator to get equal spacing.
% Watch out that the combined width of all the subfigures on a 
% line do not exceed the text width or a line break will occur.
%
%\begin{figure*}[!t]
%\centering
%\subfloat[Case I]{\includegraphics[width=2.5in]{box}%
%\label{fig_first_case}}
%\hfil
%\subfloat[Case II]{\includegraphics[width=2.5in]{box}%
%\label{fig_second_case}}
%\caption{Simulation results for the network.}
%\label{fig_sim}
%\end{figure*}
%
% Note that often IEEE papers with subfigures do not employ subfigure
% captions (using the optional argument to \subfloat[]), but instead will
% reference/describe all of them (a), (b), etc., within the main caption.
% Be aware that for subfig.sty to generate the (a), (b), etc., subfigure
% labels, the optional argument to \subfloat must be present. If a
% subcaption is not desired, just leave its contents blank,
% e.g., \subfloat[].


% An example of a floating table. Note that, for IEEE style tables, the
% \caption command should come BEFORE the table and, given that table
% captions serve much like titles, are usually capitalized except for words
% such as a, an, and, as, at, but, by, for, in, nor, of, on, or, the, to
% and up, which are usually not capitalized unless they are the first or
% last word of the caption. Table text will default to \footnotesize as
% the IEEE normally uses this smaller font for tables.
% The \label must come after \caption as always.
%
%\begin{table}[!t]
%% increase table row spacing, adjust to taste
%\renewcommand{\arraystretch}{1.3}
% if using array.sty, it might be a good idea to tweak the value of
% \extrarowheight as needed to properly center the text within the cells
%\caption{An Example of a Table}
%\label{table_example}
%\centering
%% Some packages, such as MDW tools, offer better commands for making tables
%% than the plain LaTeX2e tabular which is used here.
%\begin{tabular}{|c||c|}
%\hline
%One & Two\\
%\hline
%Three & Four\\
%\hline
%\end{tabular}
%\end{table}


% Note that the IEEE does not put floats in the very first column
% - or typically anywhere on the first page for that matter. Also,
% in-text middle ("here") positioning is typically not used, but it
% is allowed and encouraged for Computer Society conferences (but
% not Computer Society journals). Most IEEE journals/conferences use
% top floats exclusively. 
% Note that, LaTeX2e, unlike IEEE journals/conferences, places
% footnotes above bottom floats. This can be corrected via the
% \fnbelowfloat command of the stfloats package.






% conference papers do not normally have an appendix



% use section* for acknowledgment
\ifCLASSOPTIONcompsoc
  % The Computer Society usually uses the plural form
  %\section*{Acknowledgments}
\else
  % regular IEEE prefers the singular form
  %\section*{Acknowledgment}
\fi


%The authors would like to thank...





% trigger a \newpage just before the given reference
% number - used to balance the columns on the last page
% adjust value as needed - may need to be readjusted if
% the document is modified later
%\IEEEtriggeratref{8}
% The "triggered" command can be changed if desired:
%\IEEEtriggercmd{\enlargethispage{-5in}}

% references section

% can use a bibliography generated by BibTeX as a .bbl file
% BibTeX documentation can be easily obtained at:
% http://mirror.ctan.org/biblio/bibtex/contrib/doc/
% The IEEEtran BibTeX style support page is at:
% http://www.michaelshell.org/tex/ieeetran/bibtex/
%\bibliographystyle{IEEEtran}
% argument is your BibTeX string definitions and bibliography database(s)
%\bibliography{IEEEabrv,../bib/paper}
%
% <OR> manually copy in the resultant .bbl file
% set second argument of \begin to the number of references
% (used to reserve space for the reference number labels box)
\begin{thebibliography}{1}

\bibitem{aprs}
APRS Working Group, \emph{APRS Protocol Reference,} Version 1.0.1, 2000.

\end{thebibliography}




% that's all folks
\end{document}


